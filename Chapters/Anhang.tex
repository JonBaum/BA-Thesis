\chapter{Tabellen}

\section{Anzahl an Wohneinheiten nach Alter und Gebäudetyp (IWU Berechnungen)}

\begin{table}[H]\centering
\begin{tabular}{l|ccccccccc|r}
\toprule[1.5pt]
 & \multicolumn{9}{c}{Baualtersklasse} &  \\ 
\cmidrule[1.5pt]{2-10}
Gebäude-& vor 1918 & 1919 & 1949 & 1958 & 1969 & 1979 & 1984 & 1995 & 2002 & Summe \\ 
typ & & -1948&-1957&-1968&-1978&-1983&-1994&-2001&-2006&\\ \addlinespace[5pt]
\midrule[2pt]
EFH & 1.707 & 2.010 & 1.915 & 2.274 & 1.867 & 936 & 2.055 & 1.994 & 671 & 15.429 \\
\midrule
RH & 145 & 326 & 231 & 348 & 517 & 202 & 281 & 285 & 83 & 2.418 \\
\midrule
MFH & 1.501 & 2.034 & 1.912 & 2.210 & 1.677 & 821 & 1.712 & 2.240 & 296 & 14.403 \\
\midrule
GMH & 448 & 169 & 703 & 784 & 697 & -- & -- & -- & -- & 2.801 \\
\midrule
HH & -- & -- & -- & 198 & 198 & -- & -- & -- & -- & 396 \\
\midrule
MFH NBL & -- & -- & 329 & 408 & -- & -- & -- & -- & -- & 737 \\
\midrule
GMH NBL & -- & -- & -- & -- & 390 & 336 & 305 & -- & -- & 1.031 \\
\midrule
HH NBL & -- & -- & -- & -- & 310 & 67 & -- & -- & -- & 377 \\
%\midrule
%2002 - 2009 & 0,25 & 0,3 & 1,4 & 0,28 & 0,2 & 0,25 & 1,4 & 0,32 \\
%\midrule
%2010 - 2015 & 0,2 & 0,28 & 1,3 & 0,35 & 0,2 & 0,28 & 1,3 & 0,35 \\
\bottomrule[1.5pt] \addlinespace[10pt]
\end{tabular}
\caption{Anzahl an Wohneinheiten [in Tausend] nach Baualtersklasse und Gebäudetyp. \cite{Diefenbach.12.11.2007}}
\label{tab: TabelleA0}
\end{table}

EFH = Einfamilienhaus, RH = Reihenhaus, MFH = Mehrfamilienhaus, GMH = großes Mehrfamilienhaus, HH = Hochhaus, NBL = neue Bundesländer

\section{U-Werte nach TABULA}

\begin{table}[H]\centering
\begin{tabular}{lcccc|cccc}
\toprule[1.5pt]
 & \multicolumn{4}{c}{SFH}& \multicolumn{4}{c}{MFH}  \\ 
\cmidrule[1.5pt]{2-9}
Baualtersklasse & Dach & Außenwand & Fenster & Boden & Dach & Außenwand & Fenster & Boden \\ \addlinespace[5pt]
\midrule[2pt]
vor 1918 & 1,3 & 1,7 & 2,8 & 0,88 & 1,3 & 2,2 & 2,7 & 0,88 \\
\midrule
1919 - 1948 & 1,4 & 1,7 & 2,8 & 0,77 & 1,4 & 1,7 & 3 & 0,77 \\
\midrule
1949 - 1957 & 1,4 & 1,4 & 2,8 & 0,78 & 1,08 & 1,2 & 3 & 1,33 \\
\midrule
1958 - 1968 & 0,8 & 1,2 & 2,8 & 1,08 & 0,51 & 1,2 & 3 & 1,08 \\
\midrule
1969 - 1978 & 0,5 & 1 & 2,8 & 0,77 & 0,51 & 1 & 3 & 0,77 \\
\midrule
1979 - 1983 & 0,5 & 0,8 & 4,3 & 0,65 & 0,43 & 0,8 & 3 & 0,65 \\
\midrule
1984 - 1994 & 0,4 & 0,5 & 3,2 & 0,51 & 0,36 & 0,6 & 3 & 0,51 \\
\midrule
1995 - 2001 & 0,35 & 0,3 & 1,9 & 0,4 & 0,32 & 0,4 & 1,9 & 0,4 \\
\midrule
2002 - 2009 & 0,25 & 0,3 & 1,4 & 0,28 & 0,2 & 0,25 & 1,4 & 0,32 \\
\midrule
2010 - 2015 & 0,2 & 0,28 & 1,3 & 0,35 & 0,2 & 0,28 & 1,3 & 0,35 \\
\bottomrule[1.5pt] \addlinespace[10pt]
\end{tabular}
\caption{Wärmedurchgangskoeffizienten der Bauteile Dach, Außenwand, Fenster (\(U_w\)) und Boden nach Gebäudeart und Baualtersklasse [in \(W/(m^2 \cdot K)\)]}
\label{tab: TabelleA1}
\end{table}

\section{Anzahl, Anteil und \(U_g\)-Werte gängiger Verglasungsarten}

\begin{table}[H]\centering
\begin{tabular}{lccc}
\toprule[1.5pt]
Verglasungstyp & Anzahl & Anteil am Bestand & \(U_g\)-Wert \\
 & [in Millionen] & [in \%] & [in \(W/(m^2 \cdot K)\)] \\ \addlinespace[5pt]
\midrule[2pt]
Einfachverglasung & 19,6 & 3 & 5,8 \\
\midrule
Verbund- und Kastenfenster & 44,8 & 7 & 2,8 \\
\midrule
Dreischeiben-Wärmedämmglas & 48,9 & 8 & 0,7 \\
\midrule
unbeschichtetes Isolierglas & 207,3 & 34 & 2,8 \\
\midrule
Zweischeiben-Wärmedämmglas & 284,2 & 47 & 1,4 - 1,1 \\
\bottomrule[1.5pt] \addlinespace[10pt]
\end{tabular}
\caption{Bestand an Fenstern in Deutschland im Jahr 2015. \cite{Bigalke.2016}}
\label{tab: TabelleA2}
\end{table}

\chapter{Wichtiger Anhang}
