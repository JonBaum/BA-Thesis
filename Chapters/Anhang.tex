\chapter{Tabellen}


\section{U-Werte nach TABULA}

\begin{table}[H]\centering
\begin{tabular}{lcccc|cccc}
\toprule[1.5pt]
 & \multicolumn{4}{c}{SFH}& \multicolumn{4}{c}{MFH}  \\ 
\cmidrule[1.5pt]{2-9}
Baualtersklasse & Dach & Außenwand & Fenster & Boden & Dach & Außenwand & Fenster & Boden \\ \addlinespace[5pt]
\midrule[2pt]
vor 1918 & 1,3 & 1,7 & 2,8 & 0,88 & 1,3 & 2,2 & 2,7 & 0,88 \\
\midrule
1919 - 1948 & 1,4 & 1,7 & 2,8 & 0,77 & 1,4 & 1,7 & 3 & 0,77 \\
\midrule
1949 - 1957 & 1,4 & 1,4 & 2,8 & 0,78 & 1,08 & 1,2 & 3 & 1,33 \\
\midrule
1958 - 1968 & 0,8 & 1,2 & 2,8 & 1,08 & 0,51 & 1,2 & 3 & 1,08 \\
\midrule
1969 - 1978 & 0,5 & 1 & 2,8 & 0,77 & 0,51 & 1 & 3 & 0,77 \\
\midrule
1979 - 1983 & 0,5 & 0,8 & 4,3 & 0,65 & 0,43 & 0,8 & 3 & 0,65 \\
\midrule
1984 - 1994 & 0,4 & 0,5 & 3,2 & 0,51 & 0,36 & 0,6 & 3 & 0,51 \\
\midrule
1995 - 2001 & 0,35 & 0,3 & 1,9 & 0,4 & 0,32 & 0,4 & 1,9 & 0,4 \\
\midrule
2002 - 2009 & 0,25 & 0,3 & 1,4 & 0,28 & 0,2 & 0,25 & 1,4 & 0,32 \\
\midrule
2010 - 2015 & 0,2 & 0,28 & 1,3 & 0,35 & 0,2 & 0,28 & 1,3 & 0,35 \\
\bottomrule[1.5pt] \addlinespace[10pt]
\end{tabular}
\caption{Wärmedurchgangskoeffizienten der Bauteile Dach, Außenwand, Fenster (\(U_w\)) und Boden nach Gebäudeart und Baualtersklasse [in \(W/(m^2 \cdot K)\)]}
\label{tab: TabelleA1}
\end{table}

\section{Anzahl, Anteil und \(U_g\)-Werte gängiger Verglasungsarten}

\begin{table}[H]\centering
\begin{tabular}{lccc}
\toprule[1.5pt]
Verglasungstyp & Anzahl & Anteil am Bestand & \(U_g\)-Wert \\
 & [in Millionen] & [in \%] & [in \(W/(m^2 \cdot K)\)] \\ \addlinespace[5pt]
\midrule[2pt]
Einfachverglasung & 19,6 & 3 & 5,8 \\
\midrule
Verbund- und Kastenfenster & 44,8 & 7 & 2,8 \\
\midrule
Dreischeiben-Wärmedämmglas & 48,9 & 8 & 0,7 \\
\midrule
unbeschichtetes Isolierglas & 207,3 & 34 & 2,8 \\
\midrule
Zweischeiben-Wärmedämmglas & 284,2 & 47 & 1,4 - 1,1 \\
\bottomrule[1.5pt] \addlinespace[10pt]
\end{tabular}
\caption{Bestand an Fenstern in Deutschland, 2015 \cite{Bigalke.2016}}
\label{tab: TabelleA2}
\end{table}

\chapter{Wichtiger Anhang}
