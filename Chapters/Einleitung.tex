\chapter{Einleitung}

Die Folgen des Klimawandels stellt die Gesellschaft vor eine enorme Herausforderung.
Um den distopischen Prognosen entgegenzuwirken, wird der Ruf nach einem Wandel der Klimapolitik immer lauter.
Nach dem Verfehlen des europäischen Klimaschutzziels bis 2020 20\,\% weniger Kohlenstoffdioxid zu emittieren, wird der Handlungsbedarf durch die Politik bei ökologischen Themen zu einer zentralen, gesellschaftlichen Forderungen, die durch alle Generationen und vor allem von der jungen Bevölkerung ausgeht. %https://www.umweltbundesamt.de/daten/klima/klimaschutzziele-deutschlands 
Zum Erreichen des hochgesteckten Ziels der Bundesregierung, bis 2050 ein treibhausgasneutraler Staat zu sein, müssen sektorenübergreifend Maßnahmen zur Senkung des Energieverbrauchs und zur Steigerung der Energieeffizienz dürchgeführt werden. %https://www.bmu.de/themen/klima-energie/klimaschutz/nationale-klimapolitik/klimaschutzplan-2050/

Mit 665 TWh Energieverbrauch im Jahr 2016 und einem Anteil von etwa einem Viertel des gesamten deutschen Endenerieverbrauchs wird ein wichtiger Faktor zur Erfüllung der Klimaschutzziele durch den Gebäudesektor definiert.
Gut zwei Drittel der Energie wird hierbei zur Beheizung der Wohnfläche aufgewendet.
%https://www.umweltbundesamt.de/daten/private-haushalte-konsum/wohnen/energieverbrauch-privater-haushalte 
Weiterhin stellt das Potential des Wohngebäudebereichs aufgrund dessen hohen Anteil an Altbauten mit schlechten Wärmeschutzeigenschaften eine äußerst relevante Option dar, um durch Maßnahmen zur Effizienzsteigerung die CO\(_2\)-Emissionen zu reduzieren.
Als zentrales Ziel wird hierbei die Verbesserung der Sanierungsquote formuliert. 
Diese liegt derzeit bei nur 1\,\% und wird als essenzieller Hebel zur Emissionsreduktion im Wohngebäudebereich angesehen.
%https://www.umweltbundesamt.de/sites/default/files/medien/1410/publikationen/2019-06-03-barrierefrei-broschuere_wohnenundsanieren.pdf

Zur Bewertung und Verbesserung des Wohngebäudebestands werden Gebäudeenergiesysteme betrachtet.
Als Gebäudeenergiesystem werden alle Komponenten bezeichnet, die zur Energieumwandlung oder zum Energieverlust eines Gebäudes beitragen.
Neben der Anlagentechnik zur Energieerzeugung und -speicherung werde auch die vor Wärmeverlusten schützende Gebäudehülle betrachtet.
Das Hauptaugenmerk liegt auf Bauteilen mit Kontakt zur Umgebung oder zu unbeheizten Räumen, die als Bindeglied zwischen beheiztem Wohnraum und kalter Umwelt für Wärmeverluste mitverantwortlich sind. \\
Da die Sanierung im Wohngebäudesektor zum großen Teil durch Privatpersonen durchgeführt wird, spielt die Ökonomie der Maßnahmen eine wichtige Rolle.
Die Wirtschaftlichkeit wird wiederum durch staatliche Subventionen und Regulationen beeinflusst.

Als Ziel dieser Arbeit wird somit das Aufzeigen besonders wirksamer Maßnahmenpakete definiert, mit denen der deutsche Wohngebäudebestand möglichst kosteneffizient seine CO\(_2\)-Emissionen reduziert.
Hierzu werden neben ökologisch und ökonomisch wirksamen Gebäudeenergiesysteme auch staatliche Förderprogramme zur Energieeffizienzsteigerung der Wohngebäude betrachtet, um somit einen Sanierungsfahrplan für den deutschen WOhngebäudebestand zu erhalten.