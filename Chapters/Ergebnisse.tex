\chapter{Sanierungsfarhplan zur energetischen Ertüchtigung des deutschen Wohngebäudebestands}

Zur Entiwcklung einer effizienten Strategie, welche eine Minimierung der CO\(_2\)-Emission des deutschen Wohngebäudebestands bewirkt, werden in Kapitel \ref{sec:Sektion 61}  Optimierungsergebnisse für verschiedene Szenarien beschrieben.
Diese Ergebnisse werden weiterhin in \ref{sec:Sektion 62} diskutiert und bewertet.

\section{Ergebnissdarstellung}
\label{sec:Sektion 61}

Auf Basis der Eingabeparameter und Erweiterungen, welche in den Kapiteln \ref{sec:Sektion 51} und \ref{sec:Sektion 52} vorgestellt werden, bestimmt das Optimierungsprogramm optimale Gebäudeenergiesysteme.
Außerdem können dem Optimierungsprogramm verschiedene Szenarien vorgeschrieben werden, welche als Nebenbedingungen an das Modell übergeben werden und somit den zulässigen Lösungsraum beschränken.

Zunächst wird ein Referenzpunkt für die in Kapitel \ref{sec:Sektion 41} vorgestellten Gebäudeklassen bestimmt.
Hierfür werden für die in \ref{sec:Sektion 52} festgelegten Standorte eine Optimierung durchgeführt, bei welcher die Gebäudehülle nicht saniert wird und außerdem keine Anlagentechnik außer einem Kessel und einem thermischen Speicher gekauft wird.
Dieses Szenario wird als \textit{benchmark} bezeichnet und wird in TabelleXXXX vorgestellt.
Für einen Gebäudetyp und eine Baualtersklasse unterscheiden sich die Ergebnisse im Hinblick auf den Standort nur anhand der eingelesenen meteorologischen Profile.
Aus TabelleXXXX lässt sich erkennen, dass die klimatischen Unterschiede die Kosten des Energiesystems und die Emission beeiflussen. XXXX Erläutern XXXX

Als ein weiteres Szenario werden die Auswirkungen bei einer reinen Wärmeerzeugung durch Wärmepumpen untersucht.
Hierbei zieht das Optimierungsprogramm energetische Ertüchtigungen der Gebäudehülle in Betracht und die Deckung des Heizwärmebedarfs durch Wärmepumpen.
Außer einem thermischen Speicher wird die Auswahl weiterer Anlagentechnik untersagt.
XXXX Tabelle mit Ergebnissen, Ergebnisse, Elektrizitätsbedarf kurz anschneiden XXXX

Außerdem wird eine Variante untersucht, bei der die Gebäude nur BHKWs zur Energieerzeugung nutzen.
Auch in diesem Fall wird die Wahl von Sanierungsmöglichkeiten der Gebäudehülle nicht restriktiert und es steht neben den BHKWs nur thermische Speicher zur Verfügung.
XXXX Tabelle mit Ergebnissen, Ergebnisse, Elektrizitätsbedarf kurz anschneiden XXXX
















\section{Ergebnissdiskussion}
\label{sec:Sektion 62}
