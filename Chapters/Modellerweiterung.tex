\chapter{Modellerweiterung und Parameterwahl}

Aus Basis der im vorherigen Kapitel beschriebenen Parametern wird das Optimierungsprogramm um einen dynamischen Modellierungsansatz der Lüftungswärmeverluste erweitert.
Weiterhin werden Annahmen für Inputparameter getroffen und diese näher erläutert.

\section{Modellierung dynamischer Lüftungswärmeverluste}
\label{sec:Sektion 51}

In Kapitel \ref{sec:Sektion 32} wird die Berechnung der Lüftungswärmeverlust in dem Optimierungsprogramm vorgestellt.
Aus Gleichung \ref{eq:Gleichung3291} geht hervor, dass diese Verluste statisch modelliert werden.
Um den in Kapitel \ref{subsec:Sektion 331} erläuterten Rebound-Effekt zu verringern, wird die Berechnung der Lüftungswärmeverluste hin zu einem dynamischen Ansatz angepasst.
Dafür werden die Lüftungswärmeverluste in die Anteile der Fensterlüftung, der Infiltration und der maschinellen Lüftung zerlegt.

\section*{Luftvolumenstrom aus Fensterlüftung}

Zunächst werden die Lüftungswärmeverluste aufgrund eines Luftvolumenstroms durch ein geöffnetes Fenster betrachtet.
Hierfür stehen die in Tabelle \ref{tab: Tabelle3313} dargestellten Berechnungsansätze zur Verfügung.\\
Die Formel von Hall berücksichtigt keinen Einfluss des Windes und wird zur Berechnung des thermisch induzierten Luftvolumenstromes durch ein Fenster genutzt.
Dadurch wird ein minimaler Volumenstrom berechnet, welcher von einem hygienischen und bauphysikalischen Blickpunkt aus sichergestellt werden muss.
Da in dem Optimierungsmodell dieser Aspekt keine Rolle spielt und nur die Wärmeverluste von Bedeutung sind, wird auf eine Modellierung des Fensterluftvolumenstroms nach Hall verzichtet.\\
In dem Ansatz von Maas werden die Einflüsse der Temperatur und des Windes berücksichtigt.
Allerdings werden Fitkoeffizienten benutzt, welche experimentell bestimmt werden und somit vom Anwendungsfall abhängen.
Dies ist schwer in dem Optimierungsprogramm zu modellieren, weshalb der Berechnungsansatz des Fensterluftvolumenstroms nach Maas keine Berücksichtigung findet.\\
Folglich verbleibt der Berechnungsansatz der DIN EN 12831 zur Bestimmung des Luftvolumenstroms durch ein geöffnetes Fenster.
Nach Gleichung \ref{eq:Gleichung3317} und \ref{eq:Gleichung3315} werden hierzu mehrere Parameter benötigt. 
Ausgehend von einem Fenster mit 1\,m Breite und 1,2\,m Höhe ergibt sich die effektive Fensteröffnungsfläche (A\(_{eff}\)) bei einem Öffnungswinkel von 10° zu etwa 0,15\,m\(^2\).
Die meteorologischen Parameter der Temperaturdifferenz (\(\Delta \theta\)) und der Windgeschwindigkeit in 10\,m Höhe (v\(_{meteo}\)) werden über die standortabhängigen Klimaprofile eingelesen.
Für den Rauheitsparameter z\(_0\) wird eine mittlere Abschirmung angenommen, womit sich der Wert zu 0,25 ergibt.
Mit den bisher erläuterten Parametern wird der Luftvolumenstrom durch ein geöffnetes Fenster ohne Beachtung der Fassadenausrichtung berechnet.
Durch Multiplikation mit der in Kapitel \ref{sec:Sektion 42} beschriebenen Fensteröffnungsprofilen wird der gesamte Volumenstrom der Luft durch alle Fenster des Gebäudes berechnet.
Wird die Anzahl der geöffneten Fenster des Zeitpunkts t mit F(t) parametrisiert, ergibt sich der Luftvolumenstrom durch alle Fenster eines Gebäudes zu
\begin{equation}
\label{eq:Gleichung511}
q_{V,open}(t) = F(t) \cdot A_{eff} \cdot 3600 \cdot \sqrt{\frac{C_{D}^2 \cdot g \cdot h_m \cdot ( \theta_{int} - \theta_{e}(t))}{9 \cdot \theta_{e}(t)} + 0,0025 \cdot v_{fac}^2(t)} \qquad \text{.}
\end{equation}

\section*{Luftvolumenstrom aus Infiltration}

Wie in Kapitel \ref{subsec:Sektion 312} dargelegt, hängt die Infiltration von der Umgebungstemperatur, der Windgeschwindigkeit sowie der Dichtheit des Gebäudes ab.
In der Literatur konnte kein Ansatz gefunden werden, welcher mit den Eingabeparametern des Optimierungsprogramms einen Infiltrationsluftvolumenstrom unter Berücksichtigung dieser Einflussfaktoren bestimmt.
Daher erfolgt die Berechnung des Luftvolumenstroms durch Infiltration mit Hilfe der Luftwechselrate bei 50\,Pa Druckdifferenz.\\
Als Grundlage dienen die Werte nach DIN V 4108-6, welche in Tabelle \ref{tab: Tabelle3122} zu finden sind.
Hier wird die Luftdichtheit in die drei Klassen \textit{wenig dicht}, \textit{mittel dicht} und \textit{sehr dicht} eingeteilt.
Nach DIN 1946-6 hängt die Luftdichtheit des Gebäudes hauptsächlich von den Fenstern und dem Dach ab.
Daher wird die Annahme getroffen, dass sich die Luftdichheitsklassen in Abhängigkeit des gewählten Sanierungsszenarios der Fenster und des Daches darstellen lassen.
Hierbei entspricht die Klasse \textit{wenig dicht} dem Standard-Zustand der Bauteile im Bestand.
Wird mindestens eines der beiden Bauteile auf einen Standard nach EnEV 2014 saniert, wird die Dichtheitsklasse \textit{mittel dicht} angenommen.
Darüber hinaus wird die Luftdichtheit zu \textit{sehr dicht} bestimmt, wenn sowohl Dach als auch Fenster energetisch ertüchtigt werden und mindestens eines der beiden Bauteile die Vorgaben eines Passivhaus-Standard erfüllt.\\
Innerhalb der jeweiligen Dichtheitsklassen gibt die DIN V 4108-6 ein Intervall der n\(_{50}\)-Werte an.
Um den Einfluss der Gebäudedichtheit verschiedener Baujahre besser abzubilden, werden für die älteren Jahrgänge von 1860 bis 1957 die obere Grenze des Intervalls angenommen und für die jüngeren Baujahre von 1969 bis 1994 die untere Grenze. 
Der n\(_{50}\)-Wert der Gebäudealtersgruppe von 1958 bis 1968 wird mit dem Median des Intervalls abgeschätzt.
Außerdem fließen die Vorgaben der DIN 1946-6 und DIN V 4108-7 bezüglich des maximalen und minimalen Infiltrationsluftwechsel in die Annahmen des n\(_{50}\)-Wertes mit ein.
Nach Gleichung \ref{eq:Gleichung3121} wird zur Umrechnung zwischen dem Volumenstrom bei 50\,Pa Druckdifferenz und dem bei Normaldruck der Volumenstromkoeffizient e\(_Z\) benötigt.
Nach DIN 1946-6 nimmt dieser in Abhängigkeit der Geschosshöhe des Gebäudes und der Windstärke des Windgebiets Werte zwischen 0,04 und 0,09 an.
Vereinfacht wird e\(_Z\) im Rahmen des Optimierungsprogramms mit 0,05 angenommen.

\section*{Luftvolumenstrom aus maschineller Lüftung}

Da das Optimierungsprogramm die Sanierung von Altbauten betrachtet, werden die dafür vorteilhafteren dezentralen Anlagen berücksichtigt.
Weiterhin konnte in der Literatur keine Berechnung einer Kombination von natürlicher und maschineller Lüftung gefunden werden.
Eine Implementierung einer mit Kosten verbundenen maschinellen Lüftungsanlage in das Optimierungsprogramm bei gleichzeitiger Beibehaltung der Wärmeverluste aus Fensterlüftung und Infiltration ist nicht sinnvoll.
Daher wird eine bedarfsgeführte Anlage angenommen, bei welcher der Frischluftbedarf als der Luftvolumenstrom der Fensterlüftung aufgefasst wird.
Die in Tabelle \ref{tab: Tabelle3131} angegebenen Werte zur Rückwärmezahl werden hierbei konservativ abgeschätzt und weiterhin um 10\,\% gemindert, sodass sich die Rückwärmezahl zu \(\Phi\) = 60\,\%  ergibt.
Da davon auszugehen ist, dass die Nutzung einer maschinellen Lüftungsanlage die Fensterlüftung nicht ausschließt, soll durch die Abschätzung der Rückwärmezahl die Wärmerückgewinnung unter Gleichzeitiger Beachtung der Fensterlüftung abbilden.

In Abbildung XXXX ist ein Vergleich der Lüftungswärmeverluste in statischer Form, was der Berechnung aus dem Optimierungsprogramm vor Beginn der Arbeit entspricht, sowie die dynamische Modellierung nach Implementierung der zuvor genannten Volumenströme zu sehen.

XXXX Abbildung Lüftungswärmeverluste im Vergleich XXXX

\section{Wahl der Eingabeparameter zur Gebäudeenergiesystemoptimierung}
\label{sec:Sektion 52}

Wie in Kapitel \ref{sec:Sektion 32} vorgestellt, muss vor Ablauf des Optimierungsprozess das zu optimierende Gebäude parametrisiert werden.
Für den Gebäudetyp sowie das Gebäudealter wird die neue Kategorisierung aus Kapitel \ref{sec:Sektion 41} genutzt.
Hierbei wird je Altersgruppe und Gebäudetyp eine Optimierung vorgenommen.\\
Weiter wird der Standort an das Optimierungsprogramm übergeben.
Zur besseren Bewertung des klimatischen Einflusses auf das Optimierungsergebnis werden hierzu drei verschiedene Repräsentanzstationen ausgewählt.
Diese repräsentieren mit Mannheim einen warmen, mit Hamburg einen gemäßigten und mit Fichtelberg einen kalten Standort.
In Tabelle \ref{tab: Tabelle521} sind die ausgewählten Stationen mit ihren klimatologischen Angaben zu finden.
\begin{table}[H]\centering
\begin{tabular}{|l|c|c|l|}
\hline
\rowcolor[HTML]{C0C0C0} 
\cellcolor[HTML]{C0C0C0} & \multicolumn{2}{c|}{\cellcolor[HTML]{C0C0C0}Temperatur} & \cellcolor[HTML]{C0C0C0} \\ \cline{2-3}
\rowcolor[HTML]{C0C0C0} 
\multirow{-2}{*}{\cellcolor[HTML]{C0C0C0}Repräsentanzstation} & \multicolumn{1}{l|}{\cellcolor[HTML]{C0C0C0}tiefstes Monatsmittel} & \multicolumn{1}{l|}{\cellcolor[HTML]{C0C0C0}höchstes Monatsmittel} & \multirow{-2}{*}{\cellcolor[HTML]{C0C0C0}Durchlüftung} \\ \hline
Hamburg & 2,5°C & 18,1°C & gut \\ \hline
\rowcolor[HTML]{EFEFEF} 
Mannheim & 2,5°C & 20,4°C & gering \\ \hline
Fichtelberg & -3,5°C & 12,3°C & sehr gut \\ \hline
\end{tabular}
\caption{Wahl der Standorte und deren Charakteristiken bezüglich Temperatur und Durchlüftung \cite{try}}
\label{tab: Tabelle521}
\end{table}

Schließlich wird das zu optimierende Gebäude auch anhand der Wohnungsanzahl, der Wohnungsgröße und der Haushaltsgröße beschrieben.
Zu beachten ist hierbei, dass die Wohnungsanzahl für die Gebäude der Cluster A immer 1 beträgt.
Außerdem wird die anhand der Haushaltsgröße nur für Gebäude des Clusters A Energiebedarfsprofile eingelesen und die Größe besitzt für Gebäude des Clusters B keine Relevanz.\\
Zur Festlegung der Parameter Wohnungsanzahl, -größe sowie Haushaltsgröße wird sich an den durchschnittlichen Werte auf Basis des dena-Gebäudereports 2016 orientiert \cite{Bigalke.2016}.
Auf Basis dieser Daten wird der durchschnittliche Haushalt der Gebäudegruppe Cluster A zu einem Dreipersonenhaushalt festgelegt.
Weiterhin wird die durchschnittliche Wohnfläche eines Gebäudes des Clusters A zu 110\,m\(^2\) festgelegt. 
Diesen Wert erhält man aus Interpolation der Angaben aus dem dena-Gebäudereport mit der Anzahl der Wohneinheiten auf Basis der IWU-Berechnungen \cite{Diefenbach.12.11.2007}.
Für die Gebäude der Cluster B werden die Anzahl an Wohneinheiten in Intervalle eingeteilt, sodass sich die durchschnittliche Wohnungsanzahl und -größe nur schwierig zu berechnen lässt.
Hier wird als Abschätzung ein Mehrfamilienhaus mit 10 Wohnungen und je 70 m\(^2\) parametrisiert.




